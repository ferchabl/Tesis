\chapter{Introducción}

El análisis de la cobertura terrestre ha tenido un gran auge en las últimas décadas, debido a que las atividades humanas han realizado cambios drásticos a la cobertura de nuestro planeta. Por lo anterior se han puesto en marcha diversos proyectos que tienen como principal objetivo llevar a cabo el monitoreo de las coberturas globales, para los cuales se hace uso de diferentes tipos de sensores a bordo de satélites que han sido lanzados desde la década de los 80's. Entre ellos existen el sensor llamado Radiómetro de Imágenes Visibles Mediante Infrarrojos Suite (VIIRS, por sus siglas en inglés), cuyas imágenes son procesadas y distribuidas en México gracias a que la Comisión Nacional para el Conocimiento y Uso de la Biodiversidad (Conabio) cuenta con una antena y un equipo de trabajo. Las imagénes de este sensor son de gran interés ya que es relativamente nuevo en comparación con sus antecesores y por los tanto las imágenes que provee son mejores y más fáiles de manejar. 

En este trabajo se presenta la propuesta de un Algoritmo de Fusión de Imágenes Satelitales VIIRS, como alternativa a los algoritmos de Fusión que existen en la actualidad. Nuestra propuesta busca obtener resulatados similares o mejores a los obtenidos con dichos algoritmos, pero usando diferentes métodos matemáticos y representación de los datos.

La presente tesis se divide en tres capítulos. En el primer capítulo explicamos de manera breve que son las imágenes VIIRS, cual es su importancia y su uso en México; además de exponer un Algoritmo de Fusión de Imágenes Satelitales previamente desarrollado por investigadores del Centro de Investigación en Matemáticas (CIMAT). En el segundo capítulo planteamos los objetivos del algoritmo propuesto en este trabajo. Presentamos el desarrollo del mismo y su implementación en en legunaje de programción python. En el tercer y último capítulo se muestran experimentos realizados con datos reales obtenidos através de la CONABIO. El objetivo de los experimentos es hacer un anális del desempeño del nuevo método. En las conclusiones se analizan las contribuciones del trabajo desarrollado y finalmente se expone el trabajo futuro que podría realizarce como continuación de la presente tesis.