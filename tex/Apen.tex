\begin{appendices}

\section{Mínimos Cuadrados}\label{Cminimos}

\lstset{language=Python, breaklines=true, basicstyle=\footnotesize}
%\lstset{numbers=left, numberstyle=\tiny, stepnumber=1, numbersep=-2pt}
\begin{lstlisting}[frame=single]
  #Programa para ajustar un polinomio de grado 2 
  #Haciendo uso de minimos cuadrados

import numpy as np
import random as ran
import numpy.linalg

x = np.zeros(20)
y = np.zeros(20)
for i in range(0,10) : #generamos X
    x[i+10] = i
    x[i] = i-10
for i in range(0,20) :
    y[i] = pow(x[i],2) + ran.randint(1,10) #generamos Y

A = np.zeros((20,3))
for i in range(0,20) : #generamos A
    A[i,0] = 1
    A[i,1] = x[i]
    A[i,2] = pow(x[i],2)

#Calculamos el vector solucion
mc = np.dot(numpy.linalg.inv(np.dot(A.T,A)),np.dot(A.T,y))

#Calculamos los nuevos valores de Y a partir del polinimio ajustado
ny = np.zeros(20)
for i in range(0,20): 
    ny[i] = mc[0] + mc[1]*x[i] + mc[2]*pow(x[i],2)
\end{lstlisting} 

\newpage


\section{Análisis de Componentes Principales}\label{Cpca}

\lstset{language=Python, breaklines=true, basicstyle=\footnotesize}
%\lstset{numbers=left, numberstyle=\tiny, stepnumber=1, numbersep=-2pt}
\begin{lstlisting}[frame=single]
  #Programa para realizar PCA sobre un conjunto de datos X

import numpy as np 

x = base_de_datos

#Centrando a media 0
xprom = np.mean(x,1)
x = x-xprom

#Calculando SVD de X 
u,delta, vt = np.linalg.svd(xc)

v = vt.T
v = v[:4]

#Calculando la matriz de componentes principales
Y = v * xc.T

#Tomando dos componentes principales
NY = Y[:2]

#Reconstruyendo x
xcal = v.T * Y
xcal = xcal.T

\end{lstlisting}

\newpage

\section{Base de Datos Iris}\label{Iris}

A continuación se exponen los detalles de la base de datos Iris, la cual puede ser facilmente consultada en línea y para este trabajo se obtuvo de la referencia \cite{iris}

El conjunto de datos de la flor Iris es un conjunto de datos multivariante presentado por Ronald Fisher en 1936 en su artículo ``The use of multiple measurements in taxonomic problems''. 

El conjunto de datos contiene 50 muestras de cada una de las de Iris (Iris setosa, Iris virginica e Iris versicolor). Se midieron cuatro rasgos de cada muestra, lo largo y lo ancho de los sépalos y pétalos, en centímetros. 

La base de datos Iris se encuentra contenida en la siguiente tabla:

\begin{center}
\begin{tabular}{|p{2cm}|p{2cm}|p{2cm}|p{2cm}|p{2cm}|}
\hline
Largo del sépalo & Ancho del sépalo & Largo del pétalo & Ancho del pétalo & Especies \\
\hline
5.1 & 3.5 & 1.4 & 0.2 & I. setosa\\
\hline
4.9	& 3.0 & 1.4 & 0.2 & I. setosa\\
\hline
4.7	& 3.2 & 1.3 & 0.2 & I. setosa\\
\hline
4.6	& 3.1 & 1.5 & 0.2 & I. setosa\\
\hline
5.0	& 3.6 & 1.4 & 0.2 & I. setosa\\
\hline
5.4	& 3.9 & 1.7	& 0.4 & I. setosa\\
\hline
4.6	& 3.4 & 1.4	& 0.3 & I. setosa\\
\hline
5.0	& 3.4 & 1.5	& 0.2 & I. setosa\\
\hline
4.4	& 2.9 & 1.4	& 0.2 & I. setosa\\
\hline
4.9	& 3.1 & 1.5	& 0.1 & I. setosa\\
\hline
5.4	& 3.7 & 1.5	& 0.2 & I. setosa\\
\hline
4.8	& 3.4 & 1.6	& 0.2 & I. setosa\\
\hline
4.8	& 3.0 & 1.4	& 0.1 & I. setosa\\
\hline
4.3	& 3.0 & 1.1	& 0.1 & I. setosa\\
\hline
5.8	& 4.0 & 1.2	& 0.2 & I. setosa\\
\hline
5.7	& 4.4 & 1.5	& 0.4 & I. setosa\\
\hline
5.4	& 3.9 & 1.3	& 0.4 & I. setosa\\
\hline
5.1	& 3.5 & 1.4	& 0.3 & I. setosa\\
\hline
5.7	& 3.8 & 1.7	& 0.3 & I. setosa\\
\hline
\end{tabular}
\end{center}

\begin{center}
\begin{tabular}{|p{2cm}|p{2cm}|p{2cm}|p{2cm}|p{2cm}|}
\hline
Largo del sépalo & Ancho del sépalo & Largo del pétalo & Ancho del pétalo & Especies \\
\hline
5.1	& 3.8 & 1.5	& 0.3 & I. setosa\\
\hline
5.4	& 3.4 & 1.7	& 0.2 & I. setosa\\
\hline
5.1	& 3.7 & 1.5	& 0.4 & I. setosa\\
\hline
4.6	& 3.6 & 1.0	& 0.2 & I. setosa\\
\hline
5.1	& 3.3 & 1.7	& 0.5 & I. setosa\\
\hline
4.8	& 3.4 & 1.9	& 0.2 & I. setosa\\
\hline
5.0	& 3.0 & 1.6	& 0.2 & I. setosa\\
\hline
5.0	& 3.4 & 1.6	& 0.4 & I. setosa\\
\hline
5.2	& 3.5 & 1.5	& 0.2 & I. setosa\\
\hline
5.2	& 3.4 & 1.4	& 0.2 & I. setosa\\
\hline
4.7	& 3.2 & 1.6	& 0.2 & I. setosa\\
\hline
4.8	& 3.1 & 1.6	& 0.2 & I. setosa\\
\hline
5.4	& 3.4 & 1.5	& 0.4 & I. setosa\\
\hline
5.2	& 4.1 & 1.5	& 0.1 & I. setosa\\
\hline
5.5	& 4.2 & 1.4	& 0.2 & I. setosa\\
\hline
4.9	& 3.1 & 1.5	& 0.2 & I. setosa\\
\hline
5.0	& 3.2 & 1.2	& 0.2 & I. setosa\\
\hline
5.5	& 3.5 & 1.3	& 0.2 & I. setosa\\
\hline
4.9	& 3.6 & 1.4	& 0.1 & I. setosa\\
\hline
4.4	& 3.0 & 1.3	& 0.2 & I. setosa\\
\hline
5.1	& 3.4 & 1.5	& 0.2 & I. setosa\\
\hline
5.0	& 3.5 & 1.3	& 0.3 & I. setosa\\
\hline
4.5	& 2.3 & 1.3	& 0.3 & I. setosa\\
\hline
4.4	& 3.2 & 1.3	& 0.2 & I. setosa\\
\hline
5.0	& 3.5 & 1.6	& 0.6 & I. setosa\\
\hline
5.1	& 3.8 & 1.9	& 0.4 & I. setosa\\
\hline
4.8	& 3.0 & 1.4	& 0.3 & I. setosa\\
\hline
5.1	& 3.8 & 1.6	& 0.2 & I. setosa\\
\hline
4.6	& 3.2 & 1.4	& 0.2 & I. setosa\\
\hline
5.3	& 3.7 & 1.5	& 0.2 & I. setosa\\
\hline
5.0	& 3.3 & 1.4	& 0.2 & I. setosa\\
\hline
7.0	& 3.2 & 4.7	& 1.4 & I. versicolor\\
\hline
6.4	& 3.2 & 4.5	& 1.5 & I. versicolor\\
\hline
6.9	& 3.1 & 4.9	& 1.5 & I. versicolor\\
\hline
5.5	& 2.3 & 4.0	& 1.3 & I. versicolor\\
\hline
6.5	& 2.8 & 4.6	& 1.5 & I. versicolor\\
\hline
5.7	& 2.8 & 4.5	& 1.3 & I. versicolor\\
\hline
6.3	& 3.3 & 4.7	& 1.6 & I. versicolor\\
\hline
4.9	& 2.4 & 3.3	& 1.0 & I. versicolor\\
\hline
\end{tabular}
\end{center}

\begin{center}
\begin{tabular}{|p{2cm}|p{2cm}|p{2cm}|p{2cm}|p{2cm}|}
\hline
Largo del sépalo & Ancho del sépalo & Largo del pétalo & Ancho del pétalo & Especies \\
\hline
6.6	& 2.9 & 4.6	& 1.3 & I. versicolor\\
\hline
5.2	& 2.7 & 3.9	& 1.4 & I. versicolor\\
\hline
5.0	& 2.0 & 3.5	& 1.0 & I. versicolor\\
\hline
5.9	& 3.0 & 4.2	& 1.5 & I. versicolor\\
\hline
6.0	& 2.2 & 4.0	& 1.0 & I. versicolor\\
\hline
6.1	& 2.9 & 4.7	& 1.4 & I. versicolor\\
\hline
5.6	& 2.9 & 3.6	& 1.3 & I. versicolor\\
\hline
6.7	& 3.1 & 4.4	& 1.4 & I. versicolor\\
\hline
5.6	& 3.0 & 4.5	& 1.5 & I. versicolor\\
\hline
5.8	& 2.7 & 4.1	& 1.0 & I. versicolor\\
\hline
6.2	& 2.2 & 4.5	& 1.5 & I. versicolor\\
\hline
5.6	& 2.5 & 3.9	& 1.1 & I. versicolor\\
\hline
5.9	& 3.2 & 4.8	& 1.8 & I. versicolor\\
\hline
6.1	& 2.8 & 4.0	& 1.3 & I. versicolor\\
\hline
6.3	& 2.5 & 4.9	& 1.5 & I. versicolor\\
\hline
6.1	& 2.8 & 4.7	& 1.2 & I. versicolor\\
\hline
6.4	& 2.9 & 4.3	& 1.3 & I. versicolor\\
\hline
6.6	& 3.0 & 4.4	& 1.4 & I. versicolor\\
\hline
6.8	& 2.8 & 4.8	& 1.4 & I. versicolor\\
\hline
6.7	& 3.0 & 5.0	& 1.7 & I. versicolor\\
\hline
6.0	& 2.9 & 4.5	& 1.5 & I. versicolor\\
\hline
5.7	& 2.6 & 3.5	& 1.0 & I. versicolor\\
\hline
5.5	& 2.4 & 3.8	& 1.1 & I. versicolor\\
\hline
5.5	& 2.4 & 3.7	& 1.0 & I. versicolor\\
\hline
5.8	& 2.7 & 3.9	& 1.2 & I. versicolor\\
\hline
6.0	& 2.7 & 5.1	& 1.6 & I. versicolor\\
\hline
5.4	& 3.0 & 4.5	& 1.5 & I. versicolor\\
\hline
6.0	& 3.4 & 4.5	& 1.6 & I. versicolor\\
\hline
6.7	& 3.1 & 4.7	& 1.5 & I. versicolor\\
\hline
6.3	& 2.3 & 4.4	& 1.3 & I. versicolor\\
\hline
5.6	& 3.0 & 4.1	& 1.3 & I. versicolor\\
\hline
5.5	& 2.5 & 4.0	& 1.3 & I. versicolor\\
\hline
5.5	& 2.6 & 4.4	& 1.2 & I. versicolor\\
\hline
6.1	& 3.0 & 4.6	& 1.4 & I. versicolor\\
\hline
5.8	& 2.6 & 4.0	& 1.2 & I. versicolor\\
\hline
5.0	& 2.3 & 3.3	& 1.0 & I. versicolor\\
\hline
5.6	& 2.7 & 4.2	& 1.3 & I. versicolor\\
\hline
5.7	& 3.0 & 4.2	& 1.2 & I. versicolor\\
\hline
5.7	& 2.9 & 4.2	& 1.3 & I. versicolor\\
\hline
\end{tabular}
\end{center}

\begin{center}
\begin{tabular}{|p{2cm}|p{2cm}|p{2cm}|p{2cm}|p{2cm}|}
\hline
Largo del sépalo & Ancho del sépalo & Largo del pétalo & Ancho del pétalo & Especies \\
\hline
6.2	& 2.9 & 4.3	& 1.3 & I. versicolor\\
\hline
5.1	& 2.5 & 3.0	& 1.1 & I. versicolor\\
\hline
5.7	& 2.8 & 4.1	& 1.3 & I. versicolor\\
\hline
6.3	& 3.3 & 6.0	& 2.5 & I. virginica\\
\hline
5.8	& 2.7 & 5.1	& 1.9 & I. virginica\\
\hline
7.1	& 3.0 & 5.9	& 2.1 & I. virginica\\
\hline
6.3	& 2.9 & 5.6	& 1.8 & I. virginica\\
\hline
6.5	& 3.0 & 5.8	& 2.2 & I. virginica\\
\hline
7.6	& 3.0 & 6.6	& 2.1 & I. virginica\\
\hline
4.9	& 2.5 & 4.5	& 1.7 & I. virginica\\
\hline
7.3	& 2.9 & 6.3	& 1.8 & I. virginica\\
\hline
6.7	& 2.5 & 5.8	& 1.8 & I. virginica\\
\hline
7.2	& 3.6 & 6.1	& 2.5 & I. virginica\\
\hline
6.5	& 3.2 & 5.1	& 2.0 & I. virginica\\
\hline
6.4	& 2.7 & 5.3	& 1.9 & I. virginica\\
\hline
6.8	& 3.0 & 5.5	& 2.1 & I. virginica\\
\hline
5.7	& 2.5 & 5.0	& 2.0 & I. virginica\\
\hline
5.8	& 2.8 & 5.1	& 2.4 & I. virginica\\
\hline
6.4	& 3.2 & 5.3	& 2.3 & I. virginica\\
\hline
6.5	& 3.0 & 5.5	& 1.8 & I. virginica\\
\hline
7.7	& 3.8 & 6.7	& 2.2 & I. virginica\\
\hline
7.7	& 2.6 & 6.9	& 2.3 & I. virginica\\
\hline
6.0	& 2.2 & 5.0	& 1.5 & I. virginica\\
\hline
6.9	& 3.2 & 5.7	& 2.3 & I. virginica\\
\hline
5.6	& 2.8 & 4.9	& 2.0 & I. virginica\\
\hline
7.7	& 2.8 & 6.7	& 2.0 & I. virginica\\
\hline
6.3	& 2.7 & 4.9	& 1.8 & I. virginica\\
\hline
6.7	& 3.3 & 5.7	& 2.1 & I. virginica\\
\hline
7.2	& 3.2 & 6.0	& 1.8 & I. virginica\\
\hline
6.2	& 2.8 & 4.8	& 1.8 & I. virginica\\
\hline
6.1	& 3.0 & 4.9	& 1.8 & I. virginica\\
\hline
6.4	& 2.8 & 5.6	& 2.1 & I. virginica\\
\hline
7.2	& 3.0 & 5.8	& 1.6 & I. virginica\\
\hline
7.4	& 2.8 & 6.1	& 1.9 & I. virginica\\
\hline
7.9	& 3.8 & 6.4	& 2.0 & I. virginica\\
\hline
6.4	& 2.8 & 5.6	& 2.2 & I. virginica\\
\hline
6.3	& 2.8 & 5.1	& 1.5 & I. virginica\\
\hline
6.1	& 2.6 & 5.6	& 1.4 & I. virginica\\
\hline
7.7	& 3.0 & 6.1	& 2.3 & I. virginica\\
\hline
\end{tabular}
\end{center}

\begin{center}
\begin{tabular}{|p{2cm}|p{2cm}|p{2cm}|p{2cm}|p{2cm}|}
\hline
Largo del sépalo & Ancho del sépalo & Largo del pétalo & Ancho del pétalo & Especies \\
\hline
6.3	& 3.4 & 5.6	& 2.4 & I. virginica\\
\hline
6.4	& 3.1 & 5.5	& 1.8 & I. virginica\\
\hline
6.0	& 3.0 & 4.8	& 1.8 & I. virginica\\
\hline
6.9	& 3.1 & 5.4	& 2.1 & I. virginica\\
\hline
6.7	& 3.1 & 5.6	& 2.4 & I. virginica\\
\hline
6.9	& 3.1 & 5.1	& 2.3 & I. virginica\\
\hline
5.8	& 2.7 & 5.1	& 1.9 & I. virginica\\
\hline
6.8	& 3.2 & 5.9	& 2.3 & I. virginica\\
\hline
6.7	& 3.3 & 5.7	& 2.5 & I. virginica\\
\hline
6.7	& 3.0 & 5.2	& 2.3 & I. virginica\\
\hline
6.3	& 2.5 & 5.0	& 1.9 & I. virginica\\
\hline
6.5	& 3.0 & 5.2	& 2.0 & I. virginica\\
\hline
6.2	& 3.4 & 5.4	& 2.3 & I. virginica\\
\hline
5.9	& 3.0 & 5.1	& 1.8 & I. virginica\\
\hline
\end{tabular}
\end{center}

\end{appendices}
